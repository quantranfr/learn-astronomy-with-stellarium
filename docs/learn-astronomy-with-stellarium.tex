% Options for packages loaded elsewhere
\PassOptionsToPackage{unicode}{hyperref}
\PassOptionsToPackage{hyphens}{url}
%
\documentclass[
  a4paper,
]{book}
\usepackage{amsmath,amssymb}
\usepackage{iftex}
\ifPDFTeX
  \usepackage[T1]{fontenc}
  \usepackage[utf8]{inputenc}
  \usepackage{textcomp} % provide euro and other symbols
\else % if luatex or xetex
  \usepackage{unicode-math} % this also loads fontspec
  \defaultfontfeatures{Scale=MatchLowercase}
  \defaultfontfeatures[\rmfamily]{Ligatures=TeX,Scale=1}
\fi
\usepackage{lmodern}
\ifPDFTeX\else
  % xetex/luatex font selection
\fi
% Use upquote if available, for straight quotes in verbatim environments
\IfFileExists{upquote.sty}{\usepackage{upquote}}{}
\IfFileExists{microtype.sty}{% use microtype if available
  \usepackage[]{microtype}
  \UseMicrotypeSet[protrusion]{basicmath} % disable protrusion for tt fonts
}{}
\makeatletter
\@ifundefined{KOMAClassName}{% if non-KOMA class
  \IfFileExists{parskip.sty}{%
    \usepackage{parskip}
  }{% else
    \setlength{\parindent}{0pt}
    \setlength{\parskip}{6pt plus 2pt minus 1pt}}
}{% if KOMA class
  \KOMAoptions{parskip=half}}
\makeatother
\usepackage{xcolor}
\usepackage{longtable,booktabs,array}
\usepackage{calc} % for calculating minipage widths
% Correct order of tables after \paragraph or \subparagraph
\usepackage{etoolbox}
\makeatletter
\patchcmd\longtable{\par}{\if@noskipsec\mbox{}\fi\par}{}{}
\makeatother
% Allow footnotes in longtable head/foot
\IfFileExists{footnotehyper.sty}{\usepackage{footnotehyper}}{\usepackage{footnote}}
\makesavenoteenv{longtable}
\usepackage{graphicx}
\makeatletter
\def\maxwidth{\ifdim\Gin@nat@width>\linewidth\linewidth\else\Gin@nat@width\fi}
\def\maxheight{\ifdim\Gin@nat@height>\textheight\textheight\else\Gin@nat@height\fi}
\makeatother
% Scale images if necessary, so that they will not overflow the page
% margins by default, and it is still possible to overwrite the defaults
% using explicit options in \includegraphics[width, height, ...]{}
\setkeys{Gin}{width=\maxwidth,height=\maxheight,keepaspectratio}
% Set default figure placement to htbp
\makeatletter
\def\fps@figure{htbp}
\makeatother
\setlength{\emergencystretch}{3em} % prevent overfull lines
\providecommand{\tightlist}{%
  \setlength{\itemsep}{0pt}\setlength{\parskip}{0pt}}
\setcounter{secnumdepth}{5}
\usepackage{booktabs}
\ifLuaTeX
  \usepackage{selnolig}  % disable illegal ligatures
\fi
\usepackage[]{natbib}
\bibliographystyle{apalike}
\usepackage{bookmark}
\IfFileExists{xurl.sty}{\usepackage{xurl}}{} % add URL line breaks if available
\urlstyle{same}
\hypersetup{
  pdftitle={Học thiên văn qua Stellarium},
  pdfauthor={Trần Trung Quân},
  hidelinks,
  pdfcreator={LaTeX via pandoc}}

\title{Học thiên văn qua Stellarium}
\author{Trần Trung Quân}
\date{2024-04-10}

\begin{document}
\maketitle

{
\setcounter{tocdepth}{1}
\tableofcontents
}
\chapter*{Giới thiệu}\label{giux1edbi-thiux1ec7u}
\addcontentsline{toc}{chapter}{Giới thiệu}

Với những đặc thù của mình, thiên văn học là một ngành không dễ để \emph{học chay}, trong khi \emph{học qua thực hành} (learning by doing) một cách truyền thống - tức ngắm trời sao - lại gặp một số trở ngại không dễ vượt qua như địa điểm, thời tiết, thời gian, thiết bị\ldots{} Do vậy, khóa học này được tạo ra với mong muốn ai cũng có thể thực hành thiên văn học và tiếp thu được các kiến thức cơ bản về thiên văn học một cách dễ dàng nhất.

Phần lớn nội dung trong khóa học này được lấy cảm hứng từ tài liệu của Stephen Tuttle \citep{stuttle} trên trang OER Commons.

\chapter{Hệ tọa độ trời}\label{hux1ec7-tux1ecda-ux111ux1ed9-trux1eddi}

\section{Mục tiêu}\label{mux1ee5c-tiuxeau}

Hình dung được và cảm thấy tự tin khi sử dụng các hệ tọa độ trời để nói về vị trí của các thiên thể.

\section{Lý thuyết}\label{luxfd-thuyux1ebft}

Đọc phần mở đầu của trang wiki về
\href{https://vi.wikipedia.org/wiki/H\%E1\%BB\%87_t\%E1\%BB\%8Da_\%C4\%91\%E1\%BB\%99_thi\%C3\%AAn_v\%C4\%83n}{Hệ tọa độ thiên văn}.
Sau đó đọc qua các trang
\href{https://vi.wikipedia.org/wiki/H\%E1\%BB\%87_t\%E1\%BB\%8Da_\%C4\%91\%E1\%BB\%99_ch\%C3\%A2n_tr\%E1\%BB\%9Di}{Hệ tọa độ chân trời} (\emph{Horizontal} hoặc \emph{Azimuthal} hoặc \emph{Alt-Az})
và \href{https://vi.wikipedia.org/wiki/H\%E1\%BB\%87_t\%E1\%BB\%8Da_\%C4\%91\%E1\%BB\%99_x\%C3\%ADch_\%C4\%91\%E1\%BA\%A1o}{Hệ tọa độ xích đạo} (\emph{Equatorial}).

Trong khi đọc, chú ý đến các khái niệm sau:

\begin{longtable}[]{@{}lll@{}}
\toprule\noalign{}
Tiếng Việt & Tiếng Anh & Ký hiệu thường dùng \\
\midrule\noalign{}
\endhead
\bottomrule\noalign{}
\endlastfoot
Độ cao (góc cao) & Altitude & Alt \\
Góc phương vị & Azimuth & Az \\
Xích kinh & Right Ascension & \(\alpha\) \\
Xích vĩ & Declination & \(\delta\) \\
\end{longtable}

\section{Thực hành}\label{thux1ef1c-huxe0nh}

\textbf{A}. Khởi động Stellarium.

\textbf{B}. Thiết lập vị trí (\emph{Location}): ``Qui Nhon''. Quay góc nhìn về đường chân trời phía Nam (\emph{S}).

\textbf{C}. Cho thời gian dừng chạy.

\textbf{D}. Thiết lập ngày giờ: 20/3/2024, 12h30pm (giờ địa phương).

\textbf{E}. Tắt hiển thị mặt đất (\emph{Ground}) và khí quyển (\emph{Atmosphere}).

\textbf{F}. Tắt hiển thị hệ tọa độ xích đạo (\emph{Equatorial grid}); bật hiển thị hệ tọa độ chân trời (\emph{Azimuthal}). Đặt trường nhìn (\emph{FOV}) vào khoảng 60º.

\textbf{Câu hỏi 1}. Có một ngôi sao sáng ở gần đường chân trời phía Nam (±10º hai bên kinh tuyến trời). Nó tên là gì?
\_\_\_\_\_\_\_\_

\textbf{Câu hỏi 2}. Cho biết độ cao (\emph{Altitude}) của ngôi sao đó so với đường chân trời (làm tròn đến 1º, ví dụ 5º30' sẽ được làm tròn thành 6º):
\_\_º

\textbf{Câu hỏi 3}. Cho biết góc phương vị (\emph{Azimuth}) của ngôi sao này (làm tròn đến 1º):
\_\_\_º

\textbf{G}. Lùi thời gian lại 1 tháng (chỉnh ngày thành 20/2/2024, vẫn giữ nguyên giờ là 12h30pm).

\textbf{Câu hỏi 4}. Ngôi sao mà bạn xác định ở câu hỏi 1 có di chuyển so với vị trí trước đó không?
TRUE / FALSE

\textbf{Câu hỏi 5}. Giờ đây cao độ của nó là bao nhiêu (làm tròn đến 1º)?
\_\_º

\textbf{Câu hỏi 6}. Và góc phương vị của nó (làm tròn đến 1º)?
\_\_\_º

\textbf{Câu hỏi 7}. Cao độ và góc phương vị trong câu 5. và 6. có giống với kết quả của câu hỏi 2. và 3. không?
TRUE / FALSE

\begin{center}\rule{0.5\linewidth}{0.5pt}\end{center}

\textbf{H}. Chỉnh lại ngày giờ thành 20/3/2024, 12h30pm

\textbf{I}. Quay góc nhìn về đường chân trời phía Bắc (\emph{N}).

\textbf{J}. Để ý rằng sao Bắc Cực (\emph{Polaris}) rất gần với kinh tuyến trời.

\textbf{K}. Dùng chức năng tìm kiếm để tìm sao Merak. Đặt sao này vào giữa màn hình và chỉnh trường nhìn (FOV) về còn khoảng 20º.

\textbf{Câu hỏi 8}. Có một ngôi sao sáng ngay phía trên Merak. Nó tên là gì?
\_\_\_\_\_\_\_\_\_\_
và có cao độ là bao nhiêu (làm tròn đến 1º)?
\_\_\_º

\textbf{Câu hỏi 9}. Ta có thể nhìn thấy Merak và ngôi sao vừa rồi vào ngày giờ lúc đó không?
TRUE / FALSE

\begin{center}\rule{0.5\linewidth}{0.5pt}\end{center}

\textbf{L}. Chỉnh trường nhìn về khoảng 60º.

\textbf{M}. Quay góc nhìn về hướng Nam. Đảm bảo rằng chữ S nằm ở gần cạnh dưới màn hình.

\textbf{N}. Tắt hệ tọa độ chân trời (\emph{Azimuthal}) và bật hệ tọa độ xích đạo (\emph{Equatorial}).

\textbf{Câu hỏi 10}. Tìm xích kinh (\emph{Right Ascension}):
\_\_ h
\_\_ m
\_\_ s
và xích vĩ (\emph{Declination}):
\_\_\_ º
\_\_ '
của Fomalhaut.

\textbf{Câu hỏi 11}. Fomalhaut nằm trong chòm sao nào?

\begin{itemize}
\tightlist
\item
  \begin{enumerate}
  \def\labelenumi{(\Alph{enumi})}
  \tightlist
  \item
    Nam Ngư (\emph{Piscis Austrinus})\\
  \end{enumerate}
\item
  \begin{enumerate}
  \def\labelenumi{(\Alph{enumi})}
  \setcounter{enumi}{1}
  \tightlist
  \item
    Ba Giang (\emph{Eridanus}, ``ba'' = sóng)\\
  \end{enumerate}
\item
  \begin{enumerate}
  \def\labelenumi{(\Alph{enumi})}
  \setcounter{enumi}{2}
  \tightlist
  \item
    Phượng Hoàng (\emph{Phoenix})\\
  \end{enumerate}
\item
  \begin{enumerate}
  \def\labelenumi{(\Alph{enumi})}
  \setcounter{enumi}{3}
  \tightlist
  \item
    Ngọc Phu (\emph{Sculptor})
  \end{enumerate}
\end{itemize}

\textbf{O}. O. Tiến thời gian thêm một tháng (20/4/2024 lúc 12h30pm).

\textbf{Câu hỏi 12}. Fomalhaut có di chuyển so với vị trí lúc trước không?
TRUE / FALSE

\textbf{Câu hỏi 13}. Xích kinh (\emph{Right Ascension}) và xích vĩ (\emph{Declination}) của nó có thay đổi rõ rệt không?
TRUE / FALSE

\begin{center}\rule{0.5\linewidth}{0.5pt}\end{center}

\textbf{P}. Chỉnh ngày giờ về 20/3/2024 lúc 10h05'.

\textbf{Q}. Thời điểm này được chọn là do nó xấp xỉ với (thời điểm) Xuân phân (\emph{Vernal Equinox}) của năm 2024. Xích kinh (đo bằng giờ-phút-giây) được tính bắt đầu từ một điểm đặc biệt trên bầu trời, gọi là Điểm (xuân) phân (trong tiếng Anh cũng gọi là \emph{Vernal Equinox}).

\textbf{R}. Chọn Mặt Trời và đặt vào giữa màn hình. Do ta đang ở xấp xỉ thời điểm Xuân phân, Mặt Trời gần như ở giao điểm của hai đường quan trọng trên bầu trời.

\textbf{Câu hỏi 14}. Hai đường đó là?

\begin{itemize}
\tightlist
\item
  \begin{enumerate}
  \def\labelenumi{(\Alph{enumi})}
  \tightlist
  \item
    đường thẳng (\emph{Straight}) và đường chéo (\emph{Diagonal})\\
  \end{enumerate}
\item
  \begin{enumerate}
  \def\labelenumi{(\Alph{enumi})}
  \setcounter{enumi}{1}
  \tightlist
  \item
    kinh tuyến trời (\emph{Meridian}) và đường chân trời (\emph{Horizon})\\
  \end{enumerate}
\item
  \begin{enumerate}
  \def\labelenumi{(\Alph{enumi})}
  \setcounter{enumi}{2}
  \tightlist
  \item
    xích đạo trời (\emph{Celestial equator}) và hoàng đạo (\emph{Ecliptic})\\
  \end{enumerate}
\item
  \begin{enumerate}
  \def\labelenumi{(\Alph{enumi})}
  \setcounter{enumi}{3}
  \tightlist
  \item
    xích đạo trời (\emph{Celestial equator}) và kinh tuyến trời (\emph{Meridian})
  \end{enumerate}
\end{itemize}

\textbf{Câu hỏi 15}. Giao điểm của hai đường đó có xích kinh là:
\_ h
\_ m
\_ s

\section{Kết luận}\label{kux1ebft-luux1eadn}

\textbf{S}. Giờ đây bạn đã sử dụng cả hai hệ tọa độ (chân trời và xích đạo). Hãy trả lời các câu hỏi sau:

\textbf{Câu hỏi 16}. Hệ tọa độ nào tốt hơn trong việc định vị thiên thể mà không phải quan tâm đến ngày, giờ và vị trí quan sát?

\begin{itemize}
\tightlist
\item
  \begin{enumerate}
  \def\labelenumi{(\Alph{enumi})}
  \tightlist
  \item
    Hệ tọa độ chân trời (\emph{Azimuthal})\\
  \end{enumerate}
\item
  \begin{enumerate}
  \def\labelenumi{(\Alph{enumi})}
  \setcounter{enumi}{1}
  \tightlist
  \item
    Hệ tọa độ xích đạo (\emph{Equatorial})
  \end{enumerate}
\end{itemize}

\textbf{Câu hỏi 17}. Hệ tọa độ nào nên được dùng trong quan sát thiên văn?

\begin{itemize}
\tightlist
\item
  \begin{enumerate}
  \def\labelenumi{(\Alph{enumi})}
  \tightlist
  \item
    Hệ tọa độ chân trời (\emph{Azimuthal})\\
  \end{enumerate}
\item
  \begin{enumerate}
  \def\labelenumi{(\Alph{enumi})}
  \setcounter{enumi}{1}
  \tightlist
  \item
    Hệ tọa độ xích đạo (\emph{Equatorial})\\
  \end{enumerate}
\item
  \begin{enumerate}
  \def\labelenumi{(\Alph{enumi})}
  \setcounter{enumi}{2}
  \tightlist
  \item
    Còn tùy
  \end{enumerate}
\end{itemize}

\chapter{Mặt Trời và đường chân trời}\label{mux1eb7t-trux1eddi-vuxe0-ux111ux1b0ux1eddng-chuxe2n-trux1eddi}

\section{Mục tiêu}\label{mux1ee5c-tiuxeau-1}

Dự đoán được tương đối chính xác vị trí Mặt Trời mọc và lặn trên đường chân trời, dự đoán được cao độ của Mặt Trời vào giữa trưa. Vẽ được đường đi tương đối của Mặt Trời trên vòm trời trong một ngày. Từ đường đi của Mặt Trời, hiểu được nguyên do khiến ngày dài hơn vào mùa hè và ngắn hơn vào mùa đông.

\section{Thực hành}\label{thux1ef1c-huxe0nh-1}

\textbf{A}. Hiển thị công cụ Thước đo góc trên thanh công cụ bên dưới (kích hoạt plugin \emph{Angle Measure} trong menu \emph{Configuration}, tab \emph{Plugins}). Có thể cần khởi động lại Stellarium.

\textbf{B}. Chọn vị trí \emph{Qui Nhon} (phím tắt \emph{F6}). Chỉnh ngày giờ thành 20/1/2024 lúc 7h00 (phím tắt \emph{F5}). Quay hướng nhìn về phía Đông (\emph{E}), trường nhìn (\emph{FOV}) khoảng 40º.

\textbf{C}. Tắt hiệu ứng quầng chói Mặt trời (\emph{Sun's glare}) và nhật hoa (\emph{Sun's corona}) để nhìn rõ đĩa Mặt trời (phím tắt \emph{F4}, tab \emph{SSO}).

\textbf{Câu hỏi 1}. Sử dụng công cụ Thước đo góc, cho biết đường kính góc của Mặt Trời: \_\_' (zoom vào Mặt Trời, click vào Thước đo góc ở thanh công cụ bên dưới, rê chuột theo đường kính Mặt Trời).

\textbf{D}. Click vào Mặt Trời để hiện cột thông tin bên phía bên trái màn hình, nhìn vào thông tin \emph{Apparent diameter} và so sánh với kết quả vừa tìm được. Nếu khớp, chứng tỏ đĩa Mặt Trời (dù trông hơi nhỏ trên phần mềm) có kích thước đúng với thực tế.

\begin{center}\rule{0.5\linewidth}{0.5pt}\end{center}

\textbf{E}. Chọn quang cảnh là Đại dương để đường chân trời thẳng mịn (phím tắt \emph{F4}, tab \emph{Landscape}, chọn \emph{Ocean}).

\textbf{Câu hỏi 2}. Tăng giảm thời gian để tìm thời điểm mép trên của Mặt Trời chạm vào đường chân trời (định dạng \texttt{\#h\#\#}): \_\_\_\_. So sánh kết quả vừa tìm được với thông tin \emph{Rise} ở cạnh trái màn hình khi click vào Mặt Trời.

\textbf{F}. Tăng thời gian thêm 1 phút, lúc này tâm Mặt Trời gần như nằm trên đường chân trời, và số đo Cao độ (\emph{Altitude}) ở cột thông tin bên trái rất gần 0º0'.

\textbf{G}. Bật vạch chia độ đường chân trời (phím tắt \emph{F4}, tab \emph{Markings} tích chọn ô \emph{Compass marks}).

\textbf{Câu hỏi 3}. Vào ngày này, Mặt Trời không mọc vào hướng chính Đông mà mọc lệch về phía

\begin{itemize}
\tightlist
\item
  \begin{enumerate}
  \def\labelenumi{(\Alph{enumi})}
  \tightlist
  \item
    Bắc\\
  \end{enumerate}
\item
  \begin{enumerate}
  \def\labelenumi{(\Alph{enumi})}
  \setcounter{enumi}{1}
  \tightlist
  \item
    Nam
  \end{enumerate}
\end{itemize}

một góc khoảng \_\_\_\_º.

\textbf{H}. Bật hệ tọa độ xích đạo (\emph{Equatorial Grid}).

\textbf{Câu hỏi 4}. Trên đường chân trời, càng xa hướng chính Đông thì giá trị tuyệt đối của xích vĩ càng

\begin{itemize}
\tightlist
\item
  \begin{enumerate}
  \def\labelenumi{(\Alph{enumi})}
  \tightlist
  \item
    lớn\\
  \end{enumerate}
\item
  \begin{enumerate}
  \def\labelenumi{(\Alph{enumi})}
  \setcounter{enumi}{1}
  \tightlist
  \item
    nhỏ
  \end{enumerate}
\end{itemize}

Lệch về phía Bắc, xích vĩ mang giá trị

\begin{itemize}
\tightlist
\item
  \begin{enumerate}
  \def\labelenumi{(\Alph{enumi})}
  \tightlist
  \item
    dương\\
  \end{enumerate}
\item
  \begin{enumerate}
  \def\labelenumi{(\Alph{enumi})}
  \setcounter{enumi}{1}
  \tightlist
  \item
    âm
  \end{enumerate}
\end{itemize}

Còn lệch về phía Nam, xích vĩ mang giá trị

\begin{itemize}
\tightlist
\item
  \begin{enumerate}
  \def\labelenumi{(\Alph{enumi})}
  \tightlist
  \item
    dương\\
  \end{enumerate}
\item
  \begin{enumerate}
  \def\labelenumi{(\Alph{enumi})}
  \setcounter{enumi}{1}
  \tightlist
  \item
    âm
  \end{enumerate}
\end{itemize}

.

\textbf{Câu hỏi 5}. Trong ngày này, xích vĩ Mặt Trời thay đổi

\begin{itemize}
\tightlist
\item
  \begin{enumerate}
  \def\labelenumi{(\Alph{enumi})}
  \tightlist
  \item
    rõ rệt\\
  \end{enumerate}
\item
  \begin{enumerate}
  \def\labelenumi{(\Alph{enumi})}
  \setcounter{enumi}{1}
  \tightlist
  \item
    không đáng kể
  \end{enumerate}
\end{itemize}

và có giá trị khoảng \_\_\_º.

\textbf{I}. Chỉnh giờ đến 11h54 (ứng với \emph{Transit} trong cột thông tin khi click vào Mặt Trời). Đây là thời điểm Mặt Trời lên cao nhất phía trên đường chân trời.

\textbf{Câu hỏi 6}. Lúc này Mặt Trời có cao độ là bao nhiêu trong hệ tọa độ chân trời? \_\_º. Nếu không đọc thông tin từ Stellarium, liệu ta có thể suy ra cao độ này từ những dữ liệu đã biết được không (vĩ độ của Quy Nhơn: 14º, xích vĩ của Mặt Trời vào ngày này: -20º)?

\textbf{Câu hỏi 7}. Tìm thời điểm mép trên của Mặt Trời chạm vào đường chân trời phía Tây (định dạng \texttt{\#\#h\#\#}): \_\_\_\_\_. So sánh kết quả vừa tìm được với thông tin \emph{Set} ở cạnh trái màn hình khi click vào Mặt Trời.

\textbf{Câu hỏi 8}. Điền dữ liệu còn thiếu vào bảng \ref{tab:sunrise}, sử dụng cột thông tin bên trái khi click vào Mặt Trời:

\begin{longtable}[]{@{}llllll@{}}
\caption{\label{tab:sunrise} Bảng dữ liệu ở Quy Nhơn}\tabularnewline
\toprule\noalign{}
Ngày & Giờ mọc & ``Giữa trưa'' & Giờ lặn & Độ dài ngày & Xích vĩ \\
\midrule\noalign{}
\endfirsthead
\toprule\noalign{}
Ngày & Giờ mọc & ``Giữa trưa'' & Giờ lặn & Độ dài ngày & Xích vĩ \\
\midrule\noalign{}
\endhead
\bottomrule\noalign{}
\endlastfoot
20/1/2024 & \_\_\_\_ & \_\_\_\_\_ & \_\_\_\_\_ & \_\_\_\_\_ & \_\_\_º \\
20/2/2024 & \_\_\_\_ & \_\_\_\_\_ & \_\_\_\_\_ & \_\_\_\_\_ & \_\_\_º \\
20/3/2024 & \_\_\_\_ & \_\_\_\_\_ & \_\_\_\_\_ & \_\_\_\_\_ & \_\_\_º \\
20/4/2024 & \_\_\_\_ & \_\_\_\_\_ & \_\_\_\_\_ & \_\_\_\_\_ & \_\_\_º \\
20/5/2024 & 5h16 & 11h40 & 18h04 & 12h48 & 20º \\
21/6/2024 & 5h17 & 11h45 & 18h13 & 12h56 & 23º \\
21/7/2024 & 5h25 & 11h50 & 18h14 & 12h49 & 20º \\
21/8/2024 & 5h31 & 11h46 & 18h01 & 12h31 & 12º \\
22/9/2024 & 5h32 & 11h36 & 17h39 & 12h07 & 0º \\
21/10/2024 & 5h35 & 11h28 & 17h20 & 11h45 & -11º \\
21/11/2024 & 5h46 & 11h29 & 17h12 & 11h26 & -20º \\
21/12/2024 & 6h02 & 11h41 & 17h21 & 11h19 & -23º \\
\end{longtable}

\textbf{Câu hỏi 9}. Ngày dài nhất rơi vào tháng \_, ngày ngắn nhất rơi vào tháng \_\_, các tháng có ngày đêm dài gần bằng nhau: \_ và \_.

\textbf{Câu hỏi 10}. Ngày Mặt Trời \textbf{mọc sớm} nhất trùng với ngày Mặt Trời \textbf{lặn muộn} nhất: TRUE / FALSE. Ngày Mặt Trời \textbf{mọc sớm} nhất trùng với ngày dài nhất: TRUE / FALSE.

\begin{center}\rule{0.5\linewidth}{0.5pt}\end{center}

\textbf{J}. Bảng \ref{tab:seoul} trình bày các dữ liệu tương ứng ở Seoul (38º Bắc). Để ý các điểm giống và khác so với Quy Nhơn. Nhận xét về sự biến thiên của xích vĩ trong cả hai trường hợp.

\begin{longtable}[]{@{}llllll@{}}
\caption{\label{tab:seoul} Bảng dữ liệu ở Seoul}\tabularnewline
\toprule\noalign{}
Ngày & Giờ mọc & Giờ lặn & Độ dài ngày & Xích vĩ & \\
\midrule\noalign{}
\endfirsthead
\toprule\noalign{}
Ngày & Giờ mọc & Giờ lặn & Độ dài ngày & Xích vĩ & \\
\midrule\noalign{}
\endhead
\bottomrule\noalign{}
\endlastfoot
20/1/2024 & 7h44 & 17h42 & 9h58 & -20º & \\
20/2/2024 & 7h17 & 18h16 & 10h59 & -11º & \\
20/3/2024 & 6h36 & 18h44 & 12h08 & 0º & \\
20/4/2024 & 5h50 & 19h12 & 13h22 & 11º & \\
20/5/2024 & 5h19 & 19h39 & 14h20 & 20º & \\
21/6/2024 & 5h11 & 19h57 & 14h46 & 23º & \\
21/7/2024 & 5h27 & 19h49 & 14h22 & 20º & \\
21/8/2024 & 5h53 & 19h17 & 13h24 & 12º & \\
22/9/2024 & 6h20 & 18h29 & 12h09 & 0º & \\
21/10/2024 & 6h46 & 17h47 & 11h01 & -11º & \\
21/11/2024 & 7h18 & 17h17 & 9h59 & -20º & \\
21/12/2024 & 7h43 & 17h17 & 9h34 & -23º & \\
\end{longtable}

\textbf{Câu hỏi 11}. Mặt Trời mọc ở chính Đông đồng nghĩa với việc Mặt Trời mọc thẳng đứng khi nhô lên phía trên đường chân trời: TRUE / FALSE. Vậy góc tạo bởi hướng đi của Mặt Trời sau khi mọc so với đường chân trời có thể suy ra từ yếu tố nào?

\begin{itemize}
\tightlist
\item
  \begin{enumerate}
  \def\labelenumi{(\Alph{enumi})}
  \tightlist
  \item
    vĩ độ nơi quan sát\\
  \end{enumerate}
\item
  \begin{enumerate}
  \def\labelenumi{(\Alph{enumi})}
  \setcounter{enumi}{1}
  \tightlist
  \item
    kinh độ nơi quan sát
  \end{enumerate}
\end{itemize}

\section{Mở rộng}\label{mux1edf-rux1ed9ng}

Giờ Mặt Trời mọc/lặn có vẻ không biến thiên theo một quy tắc đơn giản nào. Thực vậy, ngoài việc phụ thuộc vào mùa, kinh độ, vĩ độ ở nơi quan sát, các thời điểm mọc lặn này còn phụ thuộc vào tốc độ di chuyển của Trái Đất quanh Mặt Trời: quỹ đạo của Trái Đất quanh Mặt Trời có hình ê-líp và tốc độ di chuyển không đều.

Ngoài ra, giờ Mặt Trời mọc còn phụ thuộc vào độ cao quan sát so với mực nước biển (mọc sớm hơn 1 phút khi lên cao \textasciitilde1500m). Stellarium không thể hiện được điều này.

Đọc thêm về khái niệm \href{https://en.wikipedia.org/wiki/Analemma}{Analemma} trên wikipedia.

\chapter{Quan sát Sao Thủy và Sao Kim}\label{quan-suxe1t-sao-thux1ee7y-vuxe0-sao-kim}

\section{Mục tiêu}\label{mux1ee5c-tiuxeau-2}

Sao Thủy và Sao Kim là những đối tượng quan sát thường xuyên xuất hiện trong các sự kiện ngắm sao. Qua bài thực hành này, học viên sẽ hiểu hơn về khái niệm \emph{ly giác cực đại}, biết cách xác định thời điểm quan sát lý tưởng nhất hai hành tinh trên và biết cách giải thích vị trí tương đối của chúng trên bầu trời. Học viên cũng sẽ biết cách sử dụng công cụ đo góc trong Stellarium.

\section{Lý thuyết}\label{luxfd-thuyux1ebft-1}

Đọc phần mở đầu của trang wiki về \href{https://vi.wikipedia.org/wiki/Ly_gi\%C3\%A1c_(thi\%C3\%AAn_v\%C4\%83n_h\%E1\%BB\%8Dc)}{Ly giác}

\section{Thực hành}\label{thux1ef1c-huxe0nh-2}

\textbf{A}. Hiển thị công cụ Thước đo góc (kích hoạt plugin \emph{Angle Measure} trong menu \emph{Configuration}, tab \emph{Plugins}). Có thể cần khởi động lại Stellarium.

\textbf{B}. Thiết lập vị trí (\emph{Location}): ``Qui Nhon''; ngày giờ: 10/3/2024 lúc 10:00 (giờ địa phương). Đặt thanh thời gian về phía trên bên phải màn hình.

\textbf{C}. Tắt hiển thị khí quyển (\emph{Atmosphere}).

\textbf{D}. Tắt hiển thị các ngôi sao (phím tắt \emph{F4}, tab \emph{Sky}, bỏ chọn ô \emph{Stars}).

\textbf{E}. {[}Optional{]} Tắt hiệu ứng quầng chói Mặt trời (Sun's glare) và nhật hoa (Sun's corona) để nhìn rõ đĩa Mặt trời (phím tắt \emph{F4}, tab \emph{SSO}). ``Thu nhỏ'' các hành tinh bằng cách tick chọn ô \emph{Planets} trong phần \emph{Scale}, vẫn trong menu \emph{F4} tab \emph{SSO}.

\textbf{F}. Chọn tâm điểm là Mặt Trời (menu \emph{Search}, gõ \emph{Sun}, hoặc click vào Mặt Trời và gõ dấu cách).

\begin{center}\rule{0.5\linewidth}{0.5pt}\end{center}

\textbf{G}. Tăng thời gian từng ngày một để tìm ly giác cực đại (\emph{maximum elongation}) của Sao Thủy (\emph{Mercury}). Điền ngày vào Bảng \ref{tab:elongation}.

\textbf{H}. Dùng công cụ Thước đo góc để đo khoảng cách góc giữa Mặt Trời và Sao Thủy (rê chuột) và điền vào bảng \ref{tab:elongation} (± 1º).

\textbf{I}. Tiếp tục tăng ngày cho đến khi Sao Thủy lại đạt ly giác cực đại một lần nữa, ở phía bên kia Mặt Trời. Ghi lại ngày và ly giác.

\textbf{J}. Tăng tháng và điều chỉnh ngày cho đến khi Sao Kim (\emph{Venus}) đạt ly giác cực đại ở từng phía của Mặt Trời. Điền vào Bảng \ref{tab:elongation}.

\begin{longtable}[]{@{}llllll@{}}
\caption{\label{tab:elongation} Các sự kiện ly giác cực đại}\tabularnewline
\toprule\noalign{}
Hành tinh & Hướng cực đại & Năm & Tháng & Ngày & Ly giác \\
\midrule\noalign{}
\endfirsthead
\toprule\noalign{}
Hành tinh & Hướng cực đại & Năm & Tháng & Ngày & Ly giác \\
\midrule\noalign{}
\endhead
\bottomrule\noalign{}
\endlastfoot
Sao Thủy & Đông & 2024 & 3 & \_\_ & \_\_º \\
Sao Thủy & Tây & 2024 & \_ & \_\_ & \_\_º \\
Sao Kim & Đông & \_\_\_\_ & \_ & & \_\_º \\
Sao Kim & Tây & \_\_\_\_ & \_ & & \_\_º \\
\end{longtable}

\textbf{Câu hỏi 1}. Bạn có nhận xét gì về mối quan hệ giữa các góc vừa đo được? Vào trang \href{https://in-the-sky.org/newsindex.php?feed=innerplanets&year=2024&month=3&day=10&town=1568574}{in-the-sky.org} để kiểm chứng kết quả.

\textbf{Câu hỏi 2}. Dựa vào trang web nêu trên, cho biết ngày Sao Thủy/Sao Kim đạt ly giác cực đại có nhất thiết trùng với ngày chúng xuất hiện ở vị trí cao nhất phía trên đường chân trời hay không.

\chapter{Định luật thứ 3 của Kepler}\label{ux111ux1ecbnh-luux1eadt-thux1ee9-3-cux1ee7a-kepler}

\section{Mục tiêu}\label{mux1ee5c-tiuxeau-3}

Mục tiêu đầu tiên của bài thực hành này là biết sử dụng Stellarium để quan sát được toàn bộ Hệ Mặt Trời ``từ trên cao''. Làm vậy sẽ giúp người thực hành có được hình dung chính xác về các quỹ đạo và chuyển động trong Hệ Mặt Trời. Nguyên do là trong các tài liệu phổ biến khoa học, các kích thước thường không được thể hiện đúng tỷ lệ.

Mục tiêu thứ hai là hiểu công thức và ghi nhớ Định luật thứ 3 của Kepler thông qua hoạt động tính toán kiểm chứng. Đây cũng là dịp để người thực hành tư duy về quá trình tìm ra các định luật trong tự nhiên.

\section{Lý thuyết}\label{luxfd-thuyux1ebft-2}

Đọc trang wiki \href{https://vi.wikipedia.org/wiki/C\%C3\%A1c_\%C4\%91\%E1\%BB\%8Bnh_lu\%E1\%BA\%ADt_Kepler_v\%E1\%BB\%81_chuy\%E1\%BB\%83n_\%C4\%91\%E1\%BB\%99ng_thi\%C3\%AAn_th\%E1\%BB\%83}{Các định luật của Kepler về chuyển động thiên thể}, có thể bỏ qua các công thức.

\section{Thực hành}\label{thux1ef1c-huxe0nh-3}

\textbf{A}. Khởi động Stellarium.

\textbf{B}. Thiết lập vị trí (\emph{Location}): ``Qui Nhon''. Quay góc nhìn về đường chân trời phía Nam (\emph{S}).

\textbf{C}. Cho thời gian dừng chạy. Thiết lập ngày giờ: 1/1/2024, 6am (giờ địa phương). Đặt thanh thời gian về phía trên bên phải màn hình.

\textbf{D}. Tắt hiển thị mặt đất (\emph{Ground}) và khí quyển (\emph{Atmosphere}). Hiển thị quỹ đạo của các hành tinh (menu \emph{Sky and view options}, tab \emph{SSO}, check \emph{Show orbits} + tìm hiểu ý nghĩa của các checkbox gần đó).

\begin{center}\rule{0.5\linewidth}{0.5pt}\end{center}

\textbf{E}. Nhìn tổng thể Hệ Mặt Trời (menu \emph{Location}, chọn \emph{Planet} là \emph{Solar System Observer}).

\textbf{F}. Đảm bảo rằng bạn đang nhìn thấy cực Bắc của Mặt Trời (menu \emph{Search}, gõ \emph{Sun}, bấm hết cỡ tổ hợp Alt + ↑ rồi bấm hết cỡ ↓).

\textbf{G}. Phóng to cho đến khi trường nhìn (\emph{FOV}) còn khoảng 3-4º. Ta sẽ thấy được các ``hành tinh vòng trong'' (chấm trắng + tên phía trên bên phải).

\begin{center}\rule{0.5\linewidth}{0.5pt}\end{center}

\textbf{H}. Tăng thời gian sao cho Sao Kim (\emph{Mercury}) đi được một vòng (± 1 ngày) và điền vào cột \emph{P (ngày)} và \emph{P (năm)} trong Bảng \ref{tab:kepler} (lấy số ngày chia cho 365,25).

\textbf{I}. Chỉnh lại ngày về 1/1/2024 (nếu muốn). Làm tương tự cho Sao Thủy (\emph{Venus}).

\textbf{J}. Phóng to trường nhìn lên khoảng 4-5º. Làm tương tự cho Sao Hỏa (\emph{Mars}) (± 1 tháng, lấy số tháng chia cho 12 để điền vào cột \emph{P (năm)}).

\textbf{J}. Phóng to trường nhìn lên khoảng 17-20º. Làm tương tự cho Sao Mộc (\emph{Jupiter}) và Sao Thổ (\emph{Saturn}) (± một vài tháng).

\textbf{K}. Tính bán trục lớn \emph{a} dựa theo công thức \(P^2 = a^3\)

\textbf{L}. Tính sai số so với dữ liệu chuẩn dựa theo công thức

\[\left(\frac{\text{giá trị tính được}}{\text{giá trị thực tế}} - 1\right) \times 100\%\]

\begin{longtable}[]{@{}
  >{\raggedright\arraybackslash}p{(\columnwidth - 10\tabcolsep) * \real{0.2000}}
  >{\raggedright\arraybackslash}p{(\columnwidth - 10\tabcolsep) * \real{0.1778}}
  >{\raggedright\arraybackslash}p{(\columnwidth - 10\tabcolsep) * \real{0.1556}}
  >{\raggedright\arraybackslash}p{(\columnwidth - 10\tabcolsep) * \real{0.1111}}
  >{\raggedright\arraybackslash}p{(\columnwidth - 10\tabcolsep) * \real{0.2000}}
  >{\raggedright\arraybackslash}p{(\columnwidth - 10\tabcolsep) * \real{0.1556}}@{}}
\caption{\label{tab:kepler} Bảng tổng hợp}\tabularnewline
\toprule\noalign{}
\begin{minipage}[b]{\linewidth}\raggedright
Hành tinh
\end{minipage} & \begin{minipage}[b]{\linewidth}\raggedright
P (ngày)
\end{minipage} & \begin{minipage}[b]{\linewidth}\raggedright
P (năm)
\end{minipage} & \begin{minipage}[b]{\linewidth}\raggedright
a (A.U.)
\end{minipage} & \begin{minipage}[b]{\linewidth}\raggedright
a thực tế (A.U.)
\end{minipage} & \begin{minipage}[b]{\linewidth}\raggedright
Sai số (\%)
\end{minipage} \\
\midrule\noalign{}
\endfirsthead
\toprule\noalign{}
\begin{minipage}[b]{\linewidth}\raggedright
Hành tinh
\end{minipage} & \begin{minipage}[b]{\linewidth}\raggedright
P (ngày)
\end{minipage} & \begin{minipage}[b]{\linewidth}\raggedright
P (năm)
\end{minipage} & \begin{minipage}[b]{\linewidth}\raggedright
a (A.U.)
\end{minipage} & \begin{minipage}[b]{\linewidth}\raggedright
a thực tế (A.U.)
\end{minipage} & \begin{minipage}[b]{\linewidth}\raggedright
Sai số (\%)
\end{minipage} \\
\midrule\noalign{}
\endhead
\bottomrule\noalign{}
\endlastfoot
Sao Thủy & \_\_ & \_\_\_\_ & \_\_\_\_\_ & 0.387 & \_\_\_\_ \\
Sao Kim & \_\_\_ & \_\_\_\_ & \_\_\_\_\_ & 0.723 & \_\_\_\_ \\
Trái Đất & 365.24 & 1 & 1 & 1 & 0 \\
Sao Hỏa & & \_\_\_\_ & \_\_\_\_\_ & 1.523 & \_\_\_\_ \\
Sao Mộc & & \_\_\_\_\_ & \_\_\_\_\_ & 5.204 & \_\_\_\_ \\
Sao Thổ & & \_\_\_\_\_ & \_\_\_\_\_ & 9.582 & \_\_\_\_ \\
\end{longtable}

  \bibliography{book.bib,packages.bib}

\end{document}
